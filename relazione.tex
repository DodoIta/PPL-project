\documentclass[a4paper,10pt]{article}

\usepackage[utf8]{inputenc}
\usepackage{amsmath}

\title{ \scshape{Relazione progetto PPL} }
\author{Davide Cortellucci\\\\Matricola 260321}
\date{}

\begin{document}

\maketitle							% stampa a video il titolo
\newpage							% passo ad un'altra pagina

% nuova pagina - specifica del problema
\section*{ \textbf{Specifica del problema} }			% la variabile asterisco non stampa un numero prima del titolo della sezione

Scrivere una libreria ANSI C che gestisce i numeri complessi esportando le seguenti funzioni.
% prima funzione
La prima funzione C acquisisce da tastiera un numero complesso in forma algebrica
(rispettivamente, trigonometrica) e stampa a video il numero complesso equivalente
in forma trigonometrica (rispettivamente, algebrica).
% seconda funzione
La seconda funzione C acquisisce da tastiera due numeri complessi in forma algebrica
(rispettivamente, trigonometrica) e stampa a video la loro somma in forma algebrica
(rispettivamente, trigonometrica).
% terza funzione
La terza funzione C acquisisce da tastiera due numeri complessi in forma algebrica
(rispettivamente, trigonometrica) e stampa a video la loro differenza in forma algebrica
(rispettivamente, trigonometrica).
% quarta funzione
La quarta funzione C acquisisce da tastiera due numeri complessi in forma algebrica
(rispettivamente, trigonometrica) e stampa a video il loro prodotto in forma algebrica
(rispettivamente, trigonometrica).
% quinta funzione
La quinta funzione C acquisisce da tastiera due numeri complessi in forma algebrica
(rispettivamente, trigonometrica) e stampa a video il loro quoziente in forma algebrica
(rispettivamente, trigonometrica).

\newpage

% nuova pagina - analisi del problema
\section*{ \textbf{Analisi del problema} }

\begin{enumerate}
	\item Prima funzione:
	    \begin{itemize}
	    	\item input: un numero complesso in forma algebrica (rispettivamente, trigonometrica);
		\item output: tale numero convertito in forma trigonometrica (rispettivamente, algebrica);
		\item relazioni: dato un numero $z = a + ib $ in forma algebrica,
		con $a$ parte reale e $b$ parte immaginaria, il corrispettivo in forma trigonometrica sarà un
		numero della forma $z = r[\cos(\theta) + i\sin(\theta)]$, con $r$ modulo e $\theta$ argomento,
		questi ultimi ricavabili dalle seguenti formule:\\
		$ r = \sqrt{a^2 + b^2} $\\					% formula conversione modulo
		$ \mbox{Per }\theta \in (0,2\pi]: $\\				% formula conversione argomento
		$
		\theta = 
		    \begin{cases}
		    	\frac{\pi}{2} & \mbox{ se } a = 0,\ b > 0\\
			\frac{3\pi}{2} & \mbox{ se } a = 0,\ b < 0\\
			\mbox{non definito} & \mbox{ se } a = 0,\ b = 0\\
			\arctan{\left(\frac{b}{a}\right)} & \mbox{ se } a > 0,\ b \geq 0\\
			\arctan{\left(\frac{b}{a}\right)} + 2\pi & \mbox{ se } a > 0,\ b < 0\\
			\arctan{\left(\frac{b}{a}\right)} + \pi & \mbox{ se } a < 0,\ b \mbox{ qualsiasi}
		    \end{cases}
		$
	    \end{itemize}
	\item Seconda funzione:
	    \begin{itemize}
	    	\item input: due numeri complessi in forma algebrica (rispettivamente, trigonometrica);
		\item output: somma di tali numeri in forma algebrica (rispettivamente, trigonometrica);
		\item relazioni: dati $ z_1 = a + ib $ e $ z_2 = c + id $, $ z_1 + z_2 = (a + c) + i(b + d) $
		per la forma algebrica, mentre per quella trigonometrica, dati $ z_1 = r_1[\cos(\theta_1) + i\sin(\theta_1)] $
		e $ z_2 = r_2[\cos(\theta_2) + i\sin(\theta_2)] $, $ z_1 + z_2 = [r_1\cos(\theta_1) + r_2\cos(\theta_2)] +
		i[r_1\sin(\theta_1) + r_2\sin(\theta_2)] $.
	    \end{itemize}
	\item Terza funzione:
	    \begin{itemize}
	    	\item input: due numeri complessi in forma algebrica (rispettivamente, trigonometrica);
		\item output: differenza di tali numeri in forma algebrica (rispettivamente, trigonometrica);
		\item relazioni: dati $ z_1 = a + ib $ e $ z_2 = c + id $, $ z_1 - z_2 = (a - c) + i(b - d) $
		per la forma algebrica, mentre per quella trigonometrica, dati $ z_1 = r_1[\cos(\theta_1) + i\sin(\theta_1)] $
		e $ z_2 = r_2[\cos(\theta_2) + i\sin(\theta_2)] $, $ z_1 - z_2 = [r_1\cos(\theta_1) - r_2\cos(\theta_2)] +
		i[r_1\sin(\theta_1) - r_2\sin(\theta_2)] $.
	    \end{itemize}
	\item Quarta funzione:
	    \begin{itemize}
	    	\item input: due numeri complessi in forma algebrica (rispettivamente, trigonometrica);
		\item output: prodotto di tali numeri in forma algebrica (rispettivamente, trigonometrica);
		\item relazioni: dati $ z_1 = a + ib $ e $ z_2 = c + id $, $ z_1z_2 = (ac - bd) + i(ad + bc) $
		per la forma algebrica, mentre per quella trigonometrica, dati $ z_1 = r_1[\cos(\theta_1) + i\sin(\theta_1)] $
		e $ z_2 = r_2[\cos(\theta_2) + i\sin(\theta_2)] $, $ z_1z_2 = r_1r_2[\cos(\theta_1 + \theta_2) + 
		i\sin(\theta_1 + \theta_2)] $.
	    \end{itemize}
	\item Quinta funzione:
	    \begin{itemize}
	    	\item input: due numeri complessi in forma algebrica (rispettivamente, trigonometrica);
		\item output: quoziente di tali numeri in forma algebrica (rispettivamente, trigonometrica);
		\item relazioni: dati $ z_1 = a + ib $ e $ z_2 = c + id $, $ \frac{z_1}{z_2} = (ac - bd) + i(ad + bc) $
		per la forma algebrica, mentre per quella trigonometrica, dati $ z_1 = r_1[\cos(\theta_1) + i\sin(\theta_1)] $
		e $ z_2 = r_2[\cos(\theta_2) + i\sin(\theta_2)] $, $ \frac{z_1}{z_2} = \frac{r_1}{r_2}[\cos(\theta_1 - \theta_2)
		+ i\sin(\theta_1 - \theta_2)] $.
	    \end{itemize}
\end{enumerate}

\newpage

% nuova pagina - progettazione dell'algoritmo
\section*{ \textbf{Progettazione dell'algoritmo} }

\subsection*{Scelte di progetto}
\begin{itemize}
	\item Utilizzo due strutture, una per i numeri complessi in forma algebrica ed una per quelli in forma trigonometrica.
	La prima ha un campo per la parte reale ed uno per la parte immaginaria, mentre la seconda ha un campo per il
	modulo ed uno per l'argomento.
	\item I numeri in forma algebrica sono rappresentati nella forma $ a + ib $, dove $a$ è la parte reale, $i$ è
	l'unità immaginaria e $b$ la parte immaginaria. I numeri in forma trigonometrica sono rappresentati nella forma
	$ r[\cos(\theta) + i\sin(\theta)] $, dove $r$ è il modulo e $\theta$ è l'argomento.
	\item I numeri vengono stampati con 2 cifre decimali.
	\item Si richiede all'utente di inserire l'argomento dei numeri complessi in forma trigonometrica in radianti; tale
	argomento verrà stampato a video sempre in radianti.
	\item Nella prima funzione, durante la conversione da forma algebrica a trigonometrica, i calcoli per la conversione
	tengono conto del fatto che $ \theta \in (0,2\pi] $.
	\item Nelle funzioni che calcolano la somma e la differenza di due numeri complessi, nel caso in cui essi siano 
	espressi in forma trigonometrica, tali numeri vengono convertiti in forma algebrica e ne viene computata la somma.
	\item Nella funzione che calcola il quoziente di due numeri complessi si controlla che il divisore sia diverso da zero.
	Se non lo è, la funzione comunica all'utente che la divisione non è effettuabile.
\end{itemize}

\subsection*{Passi dell'algoritmo}
\begin{enumerate}
	\item Definizione delle strutture
	\item Dichiarazione delle funzioni
	\item Prima funzione:
	    \begin{itemize}
	    	\item Inizializzazione strutture
		\item Richiesta forma (algebrica o trigonometrica) dall'utente
		\item Acquisizione numero e stampa a video
		\item Conversione e stampa a video
	    \end{itemize}
	\item Seconda funzione
	    \begin{itemize}
	    	\item Inizializzazione strutture
		\item Richiesta forma (algebrica o trigonometrica) dall'utente
		\item Acquisizione numeri e stampa a video
		\item Calcolo somma e stampa a video
	    \end{itemize}
	\item Terza funzione
	    \begin{itemize}
	    	\item Inizializzazione strutture
		\item Richiesta forma (algebrica o trigonometrica) dall'utente
		\item Acquisizione numeri e stampa a video
		\item Calcolo differenza e stampa a video
	    \end{itemize}
	\item Quarta funzione
	    \begin{itemize}
	    	\item Inizializzazione strutture
		\item Richiesta forma (algebrica o trigonometrica) dall'utente
		\item Acquisizione numeri e stampa a video
		\item Calcolo prodotto e stampa a video
	    \end{itemize}
	\item Quinta funzione
	    \begin{itemize}
	    	\item Inizializzazione strutture
		\item Richiesta forma (algebrica o trigonometrica) dall'utente
		\item Acquisizione numeri e stampa a video
		\item Controllo divisore ed eventuale calcolo del quoziente e successiva stampa a video
	    \end{itemize}
\end{enumerate}

\newpage

% nuova pagina - implementazione dell'algoritmo
\section*{ \textbf{Implementazione dell'algoritmo} }

\subsection*{File lib\_num\_complessi.c}		% trascrivo il sorgente lib_num_complessi.c
\begin{verbatim}
/* Direttive al preprocessore */

#include <stdio.h>
#include <math.h>

#define PI_GRECO 3.14

/* Definizione dei tipi */

typedef struct num_compl_alg
{
   double  parte_reale,		
           parte_imm;		
}num_compl_alg_t;

typedef struct num_compl_tri		
{
   double  modulo,
           argomento;
}num_compl_tri_t;

/* Dichiarazione delle funzioni */

void acquisisci_e_stampa(void);
void inizializza_strutt(num_compl_alg_t,
                        num_compl_tri_t);
void somma_num_compl(void);			
void inizializza_compl_alg(num_compl_alg_t,
                           num_compl_alg_t);			   			
void stampa_alg(num_compl_alg_t);		
void stampa_tri(num_compl_tri_t);		
num_compl_alg_t acquisisci_compl_alg(void);	
num_compl_tri_t acquisisci_compl_tri(void);	
void differenza_num_compl(void);		
void prodotto_num_compl(void);			
int convalida_scelta(void);			
void rapporto_num_compl(void);

/* Definizione delle funzioni */

void inizializza_strutt(num_compl_alg_t compl_1,
                        num_compl_tri_t compl_2)
{
   compl_1.parte_reale = 0;
   compl_1.parte_imm = 0;

   compl_2.modulo = 0;
   compl_2.argomento = 0;
}

/**************************************************************/

void inizializza_compl_alg(num_compl_alg_t num_1,
                           num_compl_alg_t num_2)
{
   num_1.parte_reale = 0;
   num_1.parte_imm = 0;

   num_2.parte_reale = 0;
   num_2.parte_imm = 0;
}

/**************************************************************/

int convalida_scelta(void)
{
   int esito_lettura;      /* lavoro: validazione stretta */
   char caratt_rimosso;    /* lavoro: validazione stretta */
   int scelta = 0;         /* output: scelta dell'utente */

   do                        /** VALIDAZIONE STRETTA **/
   {
      printf("Digitare 1 per la forma algebrica o 2 per
              la forma trigonometrica: ");
      esito_lettura = scanf("%d",
                            &scelta);
      if (esito_lettura != 1 || scelta < 1 || scelta > 2)
      {
         printf("Input non valido!\n");
         do
            caratt_rimosso = getchar();
            while (caratt_rimosso != '\n');
      }
   }
   while (esito_lettura != 1 || scelta < 1
          || scelta > 2);    /** FINE VALIDAZIONE STRETTA **/

   return(scelta);
}

/**************************************************************/

void stampa_alg(num_compl_alg_t numero)
{
   char parentesi_1 = '(',          /* lavoro: stampa a video */
        parentesi_2 = ')';          /* lavoro: stampa a video */
	
   /* stampo nella forma a + ib = numero complesso */
   if (numero.parte_imm >= 0)
      printf("%.2f + %.2f i.\n\n",
             numero.parte_reale,
             numero.parte_imm);
   else
      printf("%.2f + %c%.2f%c i.\n\n",
             numero.parte_reale,
             parentesi_1,
             numero.parte_imm,
             parentesi_2);
}

/**************************************************************/

void stampa_tri(num_compl_tri_t numero)
{
   /* stampo nella forma r[cos(teta) + i sin(teta)] */
   if (numero.argomento == 10)	/* argomento indefinito
                                   (a = 0, b = 0) */
      printf("indefinito (parte reale ed immaginaria
              uguali a zero!)\n\n");
   else
   {
      printf("%.2f[cos(%.2f) + i sin(%.2f)].\n",
             numero.modulo,
             numero.argomento,
             numero.argomento);
      printf("***    Notare che l'argomento e' espresso
              in radianti!    ***\n");
   }
}

/**************************************************************/

num_compl_alg_t acquisisci_compl_alg(void)
{
   int esito_lettura;       /* lavoro: validazione stretta */
   char caratt_rimosso;     /* lavoro: validazione stretta */
   num_compl_alg_t numero;  /* numero in forma algebrica */

   /* acquisisco la parte reale */
   do                           /** VALIDAZIONE STRETTA **/
   {
      printf("Inserire la parte reale: ");
      esito_lettura = scanf("%lf",
                            &numero.parte_reale);
      if (esito_lettura != 1)
      {
         printf("Input non valido!\n");
         do
            caratt_rimosso = getchar();
         while (caratt_rimosso != '\n');
      }
   }
   while (esito_lettura != 1);  /** FINE VALIDAZIONE STRETTA **/

   /* acquisisco la parte immaginaria */
   do                           /** VALIDAZIONE STRETTA **/
   {
      printf("Inserire la parte immaginaria (escludendo i,
              che verra' aggiunta automaticamente): ");
      esito_lettura = scanf("%lf",
                            &numero.parte_imm);
      if (esito_lettura != 1)
      {
         printf("Input non valido!\n");
         do
            caratt_rimosso = getchar();
         while (caratt_rimosso != '\n');
      }
   }
   while (esito_lettura != 1);  /** FINE VALIDAZIONE STRETTA **/

   return(numero);
}

/**************************************************************/

num_compl_tri_t acquisisci_compl_tri(void)
{
   int esito_lettura;         /* lavoro: validazione stretta */
   char caratt_rimosso;       /* lavoro: validazione stretta */
   num_compl_tri_t numero;    /* memorizza il numero in forma
                                 trigonometrica */

   do                           /** VALIDAZIONE STRETTA **/
   {
      printf("Inserire il modulo (si ricorda che deve essere
              reale positivo): ");
      esito_lettura = scanf("%lf",
                            &numero.modulo);
      if (esito_lettura != 1 || numero.modulo < 0)
      {
         printf("Input non valido!\n");
         do
            caratt_rimosso = getchar();
         while (caratt_rimosso != '\n');
      }
   }
   while (esito_lettura != 1 ||
          numero.modulo < 0);   /** FINE VALIDAZIONE STRETTA **/

   do                           /** VALIDAZIONE STRETTA **/
   {
      printf("Inserire l'argomento (in radianti): ");
      esito_lettura = scanf("%lf",
                            &numero.argomento);
      if (esito_lettura != 1 || numero.argomento < 0 ||
          numero.argomento > 6.28)
      {
         printf("Input non valido!\n");
         do
            caratt_rimosso = getchar();
         while (caratt_rimosso != '\n');
      }
   }
   while (esito_lettura != 1 || numero.argomento < 0 ||
          numero.argomento > 6.28);
                           /** FINE VALIDAZIONE STRETTA **/

   return(numero);
}

/**************************************************************/

void acquisisci_e_stampa(void)
{
   num_compl_alg_t forma_alg;    /* forma algebrica */
   num_compl_tri_t forma_tri;    /* forma trigonometrica */
   int scelta = 0;               /* scelta tra forma algebrica(1)
                                    e trigonometrica(2) */

   /** inizializzo le strutture **/
   inizializza_strutt(forma_alg, forma_tri);

   /** scelta della forma **/
   printf("Scegliere se si vuole esprimere il numero 
           complesso in forma algebrica o trigonometrica.\n\n");
   scelta = convalida_scelta();

   /** acquisizione della forma e relativa conversione **/
   if (scelta == 1)              /* forma algebrica */
   {
      /* acquisisco il numero dall'utente */
      forma_alg = acquisisci_compl_alg();

      /* stampo nella forma a + ib = numero complesso */
      printf("\nIl numero in forma algebrica inserito e' ");
      stampa_alg(forma_alg);

      /** conversione **/
      /* ricavo il modulo del numero */
      forma_tri.modulo = sqrt( pow(forma_alg.parte_reale, 2)
                         + pow(forma_alg.parte_imm, 2) );

      /* ricavo l'argomento del numero (distinguo 6 casi) */
      if (forma_alg.parte_reale == 0 
          && forma_alg.parte_imm > 0)       /* a = 0, b > 0 */
         forma_tri.argomento = PI_GRECO / 2;
      else if (forma_alg.parte_reale == 0
            && forma_alg.parte_imm < 0)     /* a = 0, b < 0 */
         forma_tri.argomento = 3 * (PI_GRECO / 2);
      else if (forma_alg.parte_reale == 0
               && forma_alg.parte_imm == 0) /* a = 0, b = 0 */
         forma_tri.argomento = 10;          /* 10 sta per "non
                                            definito" */
      else if (forma_alg.parte_reale > 0
               && forma_alg.parte_imm >= 0) /* a < 0, b >= 0 */
         forma_tri.argomento = atan(forma_alg.parte_imm
                               / forma_alg.parte_reale);
      else if (forma_alg.parte_reale > 0
               && forma_alg.parte_imm < 0)  /* a > 0, b < 0 */
         forma_tri.argomento = atan(forma_alg.parte_imm
                               / forma_alg.parte_reale)
                               + (2 * PI_GRECO);
      else if (forma_alg.parte_reale < 0)   /* a < 0, b assume
                                              qualsiasi valore */
         forma_tri.argomento = atan(forma_alg.parte_imm 
                               / forma_alg.parte_reale)
                               + PI_GRECO;

      /* stampo nella forma r[cos(teta) + i sin(teta)] */
      printf("L'equivalente in forma trigonometrica e' ");
      stampa_tri(forma_tri);
   }
   else                          /* forma trigonometrica */
   {
      /* acquisisco il numero dall'utente */
      forma_tri = acquisisci_compl_tri();

      /* stampo a video il numero complesso inserito
         in forma trigonometrica */
      printf("\nIl numero in forma trigonometrica inserito e' ");
      stampa_tri(forma_tri);

      /** conversione **/
      forma_alg.parte_reale = forma_tri.modulo
                              * cos(forma_tri.argomento);
      forma_alg.parte_imm = forma_tri.modulo
                            * sin(forma_tri.argomento);

      /* stampo nella forma a + ib = numero complesso */
      printf("L'equivalente in forma algebrica e' ");
      stampa_alg(forma_alg);
   }
}/*** fine funzione acquisisci_e_stampa ***/

/**************************************************************/

void somma_num_compl(void)
{
   num_compl_alg_t alg_1,     /* forma algebrica */
                   alg_2,     /* forma algebrica */
   somma_alg;                 /* somma di due numeri
                                 in forma algebrica */
   num_compl_tri_t tri_1,     /* forma trigonometrica */
                   tri_2;     /* forma trigonometrica */
   num_compl_alg_t addendo_1, /* forma trigonometrica
                                 convertito in forma
                                 algebrica */
                   addendo_2, /* forma trigonometrica
                                 convertito in forma
                                 algebrica */
                   somma_tri; /* somma di due numeri compl.
                                 in forma trigonometrica */
   int scelta = 0;            /* scelta tra forma algebrica
                                 (1) e trigonometrica(2) */

   /** inizializzo le strutture **/
   inizializza_strutt(alg_1, tri_1);
   inizializza_strutt(alg_2, tri_2);
   inizializza_compl_alg(addendo_1, addendo_2);
   inizializza_compl_alg(somma_alg, somma_tri);

   /** scelta della forma **/
   printf("Scegliere se si vuole effettuare la somma tra numeri
           complessi in forma algebrica o trigonometrica.\n");
   scelta = convalida_scelta();

   if (scelta == 1)              /* forma algebrica */
   {
      /** acquisizione della forma algebrica e relativa somma **/
      printf("Primo addendo...\n");
      alg_1 = acquisisci_compl_alg();
      printf("Secondo addendo...\n");
      alg_2 = acquisisci_compl_alg();

      /* stampo a video il primo addendo */
      printf("\nIl primo addendo inserito e' ");
      stampa_alg(alg_1);

      /* stampo a video il secondo addendo */
      printf("Il secondo addendo inserito e' ");
      stampa_alg(alg_2);

      /* calcolo la somma in forma algebrica e la stampo */
      somma_alg.parte_reale = alg_1.parte_reale
                              + alg_2.parte_reale;
      somma_alg.parte_imm = alg_1.parte_imm
                            + alg_2.parte_imm;

      printf("La somma dei numeri complessi inseriti e' ");
      stampa_alg(somma_alg);
   }
   else                          /* forma trigonometrica */
   {
         /** acquisizione della forma trigonometrica **/
      printf("Primo addendo...\n");
      tri_1 = acquisisci_compl_tri();
      printf("Secondo addendo...\n");
      tri_2 = acquisisci_compl_tri();

      /* stampo a video il primo addendo */
      printf("\nIl primo addendo inserito e' ");
      stampa_tri(tri_1);

      /* stampo a video il secondo addendo */
      printf("Il secondo addendo inserito e' ");
      stampa_tri(tri_2);

      /* calcolo somma in forma trigonometrica e stampo */
         /** conversione addendi in forma algebrica **/
      addendo_1.parte_reale = tri_1.modulo
                              * cos(tri_1.argomento);
      addendo_1.parte_imm = tri_1.modulo
                            * sin(tri_1.argomento);

      addendo_2.parte_reale = tri_2.modulo
                              * cos(tri_2.argomento);		
      addendo_2.parte_imm = tri_2.modulo
                              * sin(tri_2.argomento);		

         /** calcolo della somma in forma algebrica **/
      somma_tri.parte_reale = addendo_1.parte_reale
                              + addendo_2.parte_reale;
      somma_tri.parte_imm = addendo_1.parte_imm
                            + addendo_2.parte_imm;

      printf("La somma dei numeri complessi inseriti e' ");
      stampa_alg(somma_tri);
   }
}/*** fine funzione somma_num_compl ***/

/**************************************************************/

void differenza_num_compl(void)
{
   num_compl_alg_t minuendo_alg,   /* forma algebrica */
                   sottraendo_alg, /* forma algebrica */
                   differenza_alg; /* differenza di due numeri
                                      in forma algebrica */
   num_compl_tri_t minuendo_tri,   /* forma trigonometrica */
                   sottraendo_tri; /* forma trigonometrica */
   num_compl_alg_t operando_1,     /* forma trigonometrica
                                      convertito in forma
                                      algebrica */
                   operando_2,     /* forma trigonometrica
                                      convertito in forma
                                      algebrica */
                   differenza_tri; /* somma di due numeri compl.
                                      in forma trigonometrica */
   int scelta = 0;                 /* scelta tra forma algebrica
                                      (1) e trigonometrica(2) */

   /** inizializzo le strutture **/
   inizializza_strutt(minuendo_alg, minuendo_tri);
   inizializza_strutt(sottraendo_alg, sottraendo_tri);
   inizializza_compl_alg(operando_1, operando_2);
   inizializza_compl_alg(differenza_alg, differenza_tri);

   /** scelta della forma **/
   printf("Scegliere se si vuole effettuare la differenza tra
           numeri complessi in forma algebrica
           o trigonometrica.\n");
   scelta = convalida_scelta();

   if (scelta == 1)              /* forma algebrica */
   {
      /** acquisizione della forma algebrica e relativa diff. **/
      printf("Minuendo...\n");
      minuendo_alg = acquisisci_compl_alg();
      printf("Sottraendo...\n");
      sottraendo_alg = acquisisci_compl_alg();

      /* stampo a video il minuendo */
      printf("\nIl minuendo inserito e' ");
      stampa_alg(minuendo_alg);

      /* stampo a video il sottraendo */
      printf("Il sottraendo inserito e' ");
      stampa_alg(sottraendo_alg);

      /* calcolo la differenza in forma algebrica e la stampo */
      differenza_alg.parte_reale = minuendo_alg.parte_reale
                                   - sottraendo_alg.parte_reale;
      differenza_alg.parte_imm = minuendo_alg.parte_imm
                                 - sottraendo_alg.parte_imm;

      printf("La differenza dei numeri complessi inseriti e' ");
      stampa_alg(differenza_alg);
   }
   else                          /* forma trigonometrica */
   {
         /** acquisizione della forma trigonometrica **/
      printf("Minuendo...\n");
      minuendo_tri = acquisisci_compl_tri();
      printf("Sottraendo...\n");
      sottraendo_tri = acquisisci_compl_tri();

      /* stampo a video il minuendo */
      printf("\nIl minuendo inserito e' ");
      stampa_tri(minuendo_tri);

      /* stampo a video il sottraendo */
      printf("Il sottraendo inserito e' ");
      stampa_tri(sottraendo_tri);

      /* calcolo la diff. in forma trigonometrica e stampo */
         /** conversione operandi in forma algebrica **/
      operando_1.parte_reale = minuendo_tri.modulo
                               * cos(minuendo_tri.argomento);
      operando_1.parte_imm = minuendo_tri.modulo
                             * sin(minuendo_tri.argomento);

      operando_2.parte_reale = sottraendo_tri.modulo
                               * cos(sottraendo_tri.argomento);
      operando_2.parte_imm = sottraendo_tri.modulo
                             * sin(sottraendo_tri.argomento);

         /** calcolo diff. in forma algebrica e stampa **/
      differenza_tri.parte_reale = operando_1.parte_reale
                                   - operando_2.parte_reale;
      differenza_tri.parte_imm = operando_1.parte_imm
                                 - operando_2.parte_imm;

      printf("La differenza dei numeri complessi inseriti e' ");
      stampa_alg(differenza_tri);
   }
}
/**************************************************************/

void prodotto_num_compl(void)
{
   num_compl_alg_t moltiplicando_alg,  /* forma algebrica */
                   moltiplicatore_alg, /* forma algebrica */
                   prodotto_alg;       /* forma algebrica */
   num_compl_tri_t moltiplicando_tri,  /* forma trig. */
                   moltiplicatore_tri, /* forma trig. */
                   prodotto_tri;       /* forma trig. */
   int scelta = 0;                     /* scelta tra forma 
                                          algebrica(1) e 
                                          trigonometrica(2) */

   /** inizializzo le strutture **/
   inizializza_strutt(moltiplicando_alg, moltiplicando_tri);
   inizializza_strutt(moltiplicatore_alg, moltiplicatore_tri);
   inizializza_strutt(prodotto_alg, prodotto_tri);

   /** scelta della forma **/
   printf("Scegliere se si vuole effettuare il prodotto tra
           numeri complessi in forma algebrica o 
           trigonometrica.\n");
   scelta = convalida_scelta();

   if (scelta == 1)		/* forma algebrica */
   {
          /** acquisizione della forma algebrica e prodotto **/
      printf("Moltiplicando...\n");
      moltiplicando_alg = acquisisci_compl_alg();
      printf("Moltiplicatore...\n");
      moltiplicatore_alg = acquisisci_compl_alg();

      /* stampo a video il moltiplicando */
      printf("\nIl moltiplicando inserito e' ");
      stampa_alg(moltiplicando_alg);

      /* stampo a video il moltiplicatore */
      printf("Il moltiplicatore inserito e' ");
      stampa_alg(moltiplicatore_alg);

      /* calcolo prodotto in forma algebrica e stampo */
      prodotto_alg.parte_reale = (moltiplicando_alg.parte_reale 
                               * moltiplicatore_alg.parte_reale) 
                               - (moltiplicando_alg.parte_imm
                               * moltiplicatore_alg.parte_imm);
      prodotto_alg.parte_imm = (moltiplicando_alg.parte_reale 
                             * moltiplicatore_alg.parte_imm)
                             + (moltiplicando_alg.parte_imm
			     * moltiplicatore_alg.parte_reale);

      printf("Il prodotto dei numeri complessi inseriti e' ");
      stampa_alg(prodotto_alg);
   }
   else				/* forma trigonometrica */
   {
         /** acquisizione della forma trigonometrica **/
      printf("Moltiplicando...\n");
      moltiplicando_tri = acquisisci_compl_tri();
      printf("Moltiplicatore...\n");
      moltiplicatore_tri = acquisisci_compl_tri();

      /* stampo a video il moltiplicando */
      printf("\nIl moltiplicando inserito e' ");
      stampa_tri(moltiplicando_tri);

      /* stampo a video il moltiplicatore */
      printf("Il moltiplicatore inserito e' ");
      stampa_tri(moltiplicatore_tri);

      /* calcolo prodotto in forma trig. e stampo */
      prodotto_tri.modulo = moltiplicando_tri.modulo
                            * moltiplicatore_tri.modulo;
      prodotto_tri.argomento = moltiplicando_tri.argomento
                               + moltiplicatore_tri.argomento;

      printf("Il prodotto dei numeri complessi inseriti e' ");
      stampa_tri(prodotto_tri);
   }
}/*** fine funzione prodotto_num_compl ***/

/**************************************************************/

void rapporto_num_compl(void)
{
   num_compl_alg_t dividendo_alg,     /* forma algebrica */
                   divisore_alg,      /* forma algebrica */
                   rapporto_alg;      /* forma algebrica */
   num_compl_tri_t dividendo_tri,     /* forma trigonometrica */
                   divisore_tri,      /* forma trigonometrica */
                   rapporto_tri;      /* forma trigonometrica */
   int scelta = 0;                    /* scelta tra forma 
                                         algebrica(1) e 
                                         trigonometrica(2) */

   /** inizializzo le strutture **/
   inizializza_strutt(dividendo_alg, dividendo_tri);
   inizializza_strutt(divisore_alg, divisore_tri);
   inizializza_strutt(rapporto_alg, rapporto_tri);

   /** scelta della forma **/
   printf("Scegliere se si vuole effettuare il rapporto tra
           numeri complessi in forma algebrica
           o trigonometrica.\n");
   scelta = convalida_scelta();

   if (scelta == 1)              /* forma algebrica */
   {
         /** acquisizione della forma algebrica e rapporto **/
      printf("Dividendo...\n");
      dividendo_alg = acquisisci_compl_alg();
      printf("Divisore...\n");
      divisore_alg = acquisisci_compl_alg();

      /* controllo che il denominatore sia diverso da zero */
      if (divisore_alg.parte_reale == 0 &&
          divisore_alg.parte_imm == 0) /* se denominatore = 0 */
         printf("Il divisore inserito e' uguale a zero,
	         pertanto la divisione non e' possibile.\n");
      else       /* altrimenti proseguo e calcolo il rapporto */
      {
         /* stampo a video il dividendo */
         printf("\nIl dividendo inserito e' ");
         stampa_alg(dividendo_alg);
	
         /* stampo a video il divisore */
         printf("Il divisore inserito e' ");
         stampa_alg(divisore_alg);

         /* calcolo rapporto in forma algebrica e stampo */
         rapporto_alg.parte_reale = ( (dividendo_alg.parte_reale 
                                  * divisore_alg.parte_reale)
                                  + (dividendo_alg.parte_imm
                                  * divisore_alg.parte_imm) )
                                  / (pow(divisore_alg.parte_reale 
                                          , 2)
                                  + pow(divisore_alg.parte_imm
                                        , 2));
         rapporto_alg.parte_imm = ( (dividendo_alg.parte_imm 
                                * divisore_alg.parte_reale)
                                - (dividendo_alg.parte_reale
                                * divisore_alg.parte_imm) )
                                / ( pow(divisore_alg.parte_reale,
                                        2)
                                + pow(divisore_alg.parte_imm,
                                      2) );
	
         printf("Il rapporto dei numeri complessi inseriti e' ");
         stampa_alg(rapporto_alg);
      }
   }
   else	                         /* forma trigonometrica */
   {
         /** acquisizione della forma trigonometrica **/
      printf("Dividendo...\n");
      dividendo_tri = acquisisci_compl_tri();
      printf("Divisore...\n");
      divisore_tri = acquisisci_compl_tri();

      /* controllo che il denominatore sia diverso da zero */
      if (divisore_tri.modulo == 0)     /* se divisore = 0 */
         printf("Il divisore inserito e' uguale a zero,
                 pertanto la divisione non e' possibile.\n");
      else
      {
         /* stampo a video il dividendo */
         printf("\nIl dividendo inserito e' ");
         stampa_tri(dividendo_tri);
	
         /* stampo a video il divisore */
         printf("Il divisore inserito e' ");
         stampa_tri(divisore_tri);
	
         /* calcolo rapporto in forma trigonometrica e stampo */
         rapporto_tri.modulo = dividendo_tri.modulo
                               / divisore_tri.modulo;
         rapporto_tri.argomento = dividendo_tri.argomento
                                  - divisore_tri.argomento;
	
         printf("Il rapporto dei numeri complessi inseriti e' ");
         stampa_tri(rapporto_tri);
      }
   }
}/*** fine funzione rapporto_num_compl ***/

/**************************************************************/
\end{verbatim}						% fine del sorgente lib_num_complessi.c

\subsection*{File lib\_num\_complessi.h}		% trascrivo il sorgente lib_num_complessi.h
\begin{verbatim}
/* ridefinizione delle costanti simboliche */

#define PI_GRECO 3.14

/* ridefinizione dei tipi */

typedef struct num_compl_alg
{
   double  parte_reale,		
           parte_imm;		
}num_compl_alg_t;

typedef struct num_compl_tri		
{
   double  modulo,
           argomento;
}num_compl_tri_t;

/* ridichiarazione delle funzioni da esportare */

extern void acquisisci_e_stampa(void);   /* acquisizione */
extern void somma_num_compl(void);       /* somma */
extern void differenza_num_compl(void);	 /* differenza */
extern void prodotto_num_compl(void);    /* prodotto */
extern void rapporto_num_compl(void);    /* divisione */
\end{verbatim}						% fine del sorgente lib_num_complessi.h

\subsection*{Makefile}					% trascrivo il Makefile
\begin{verbatim}
progetto_2016: lib_num_complessi.o progetto_2016.o Makefile
   gcc -ansi -Wall -O lib_num_complessi.o progetto_2016.o
   -o progetto_2016 -lm

lib_num_complessi.o: lib_num_complessi.c 
                     lib_num_complessi.h Makefile
   gcc -ansi -Wall -O -c lib_num_complessi.c

progetto_2016.o: progetto_2016.c lib_num_complessi.h Makefile
   gcc -ansi -Wall -O -c progetto_2016.c

pulisci:
   rm -f lib_num_complessi.o progetto_2016.o

pulisci_tutto:
   rm -f progetto_2016 lib_num_complessi.o progetto_2016.o
\end{verbatim}						% fine del Makefile

\subsection*{File progetto\_2016.c}			% trascrivo il sorgente progetto_2016.c
\begin{verbatim}
/* direttive al preprocessore */

#include <stdio.h>
#include "lib_num_complessi.h"

/* dichiarazione delle funzioni */
int acquisisci_scelta(void);   /* acquisisce il valore scelta */

/* definizione delle funzioni */
int acquisisci_scelta(void)
{
   int esito_lettura;          /* lavoro: validazione stretta */
   char caratt_rimosso;        /* lavoro: validazione stretta */
   int scelta = 0;             /* output: scelta dell'utente */

   do                           /** VALIDAZIONE STRETTA **/
   {
      esito_lettura = scanf("%d",
                            &scelta);
      if (esito_lettura != 1 || scelta < 1 || scelta > 5)
      {
         printf("Input non valido!\n");
         do
            caratt_rimosso = getchar();
         while (caratt_rimosso != '\n');
      }
   }
   while (esito_lettura != 1 || scelta < 1
          || scelta > 5);       /** FINE VALIDAZIONE STRETTA **/

   return(scelta);
}

/*** MAIN ***/

int main(void)
{
   int scelta = 0;              /* scelta dell'utente */
	
   printf("\t**********************************************\n");
   printf("\t** Programma-test per "lib_num_complessi.h" **\n");
   printf("\t**********************************************\n");
   printf("\nDigitare uno dei seguenti numeri, ciascuno 
           corrispondente alla funzione indicata:\n");
   printf("1 - per convertire un numero complesso dalla forma
           algebrica alla trigonometrica e viceversa;\n");
   printf("2 - per calcolare la somma tra due numeri
           complessi;\n");
   printf("3 - per calcolare la differenza tra due numeri
           complessi;\n");
   printf("4 - per calcolare il rapporto tra due numeri
           complessi;\n");
   printf("5 - per calcolare il quoziente tra due numeri 
           complessi.\n...");

   /* acquisizione e convalida della scelta */
   scelta = acquisisci_scelta();

   switch (scelta)
   {
      case 1:
         acquisisci_e_stampa();
         break;
      case 2:
         somma_num_compl();
         break;
      case 3:
         differenza_num_compl();
         break;
      case 4:
         prodotto_num_compl();
         break;
      case 5:
         rapporto_num_compl();
         break;
      default:
         printf("Si e' verificato un errore.\n");
         break;
   }

   printf("\tEsecuzione terminata!\n");

   return(0);
}
\end{verbatim}						% fine del file progetto_2016.c

\newpage

% nuova pagina - testing del programma
\section*{ \textbf{Testing del programma} }

\begin{itemize}
    \item Conversione da forma algebrica a trigonometrica (prima funzione)\\
    	\begin{tabular}{|*3{c|}}
	\hline
			& \textbf{Forma algebrica}	& \textbf{Forma trigonometrica}\\
	\hline
	1		& $-8.00 + 0.00 i$		& $8.00[\cos{\left(3.14\right)} + i\sin{\left(3.14\right)}]$\\
	\hline
	2		& $0.00 + 0.00 i$		& indefinito\\
	\hline
	\end{tabular}
    \item Conversione da forma trigonometrica ad algebrica (prima funzione)\\
    	\begin{tabular}{|*3{c|}}
	\hline
			& \textbf{Forma trigonometrica}					& \textbf{Forma algebrica}\\
	\hline
	3		& $2.00[\cos{\left(1.57\right)} + i\sin{\left(1.57\right)}]$	& $0.00 + 2.00 i$\\
	\hline
	4		& $2.00[\cos{\left(0\right)} + i\sin{\left(0\right)}]$		& $2.00 + 0.00 i$\\
	\hline
	\end{tabular}

    \item Somma algebrica (seconda funzione)\\
    	\begin{tabular}{|*3{c|}}
	\hline
	Primo addendo			& Secondo addendo		& Somma\\
	\hline
	$0.00 + 5.00 i$			& $1.00 + 2.00 i$		& $1.00 + 7.00 i$\\ 
	\hline
	$2.00 + 11.00 i$		& $-4.00 + 0.00 i$		& $-2.00 + 11.00 i$\\
	\hline
	\end{tabular}

    \item Somma trigonometrica (seconda funzione)\\
    	\begin{tabular}{|*2{c|}}
	\hline
	\bfseries Primo addendo						& \bfseries Secondo addendo\\
	\hline
	$4.00[\cos{\left(3.14\right)} + i\sin{\left(3.14\right)}]$	& $3.00[\cos{\left(4.71\right)} + i\sin{\left(4.71\right)}]$\\
	\hline
	\multicolumn{2}{|c|}{\textbf{Somma}}\\
	\hline
	\multicolumn{2}{|c|}{$-4.01 + (-2.99)i$}\\
	\hline
	\end{tabular}

    \item Differenza algebrica (terza funzione)\\
    	\begin{tabular}{|*3{c|}}
	\hline
	\bfseries Minuendo		& \bfseries Sottraendo		& \bfseries Differenza\\
	\hline
	$0.00 + 0.00 i$			& $-1.00 + 5.00 i$		& $1.00 + (-5.00) i$\\
	\hline
	\end{tabular}

    \item Differenza trigonometrica (terza funzione)\\
    	\begin{tabular}{|*2{c|}}
	\hline
	\bfseries Minuendo						& \bfseries Sottraendo\\
	\hline
	$3.16[\cos{\left(1.25\right)} + i\sin{\left(1.25\right)}]$	& $4.47[\cos{\left(5.82\right)} + i\sin{\left(5.82\right)}]$\\
	\hline
	\multicolumn{2}{|c|}{\textbf{Differenza}}\\
	\hline
	\multicolumn{2}{|c|}{$-3.00 + 5.00 i$}\\
	\hline
	\end{tabular}

    \item Prodotto algebrico (quarta funzione)\\
    	\begin{tabular}{|*3{c|}}
	\hline
	\bfseries Moltiplicando		& \bfseries Moltiplicatore	& \bfseries Prodotto\\
	\hline
	$1.00 + 1.00 i$			& $1.00 + 3.00 i$		& $-2.00 + 4.00 i$\\
	\hline
	\end{tabular}

    \item Prodotto trigonometrico (quarta funzione)\\
    	\begin{tabular}{|*2{c|}}
	\hline
	\bfseries Moltiplicando						& \bfseries Moltiplicatore\\
	\hline
	$8.00[\cos{\left(3.14\right)} + i\sin{\left(3.14\right)}]$	& $2.00[\cos{\left(1.57\right)} + i\sin{\left(1.57\right)}]$\\
	\hline
	\multicolumn{2}{|c|}{\textbf{Prodotto}}\\
	\hline
	\multicolumn{2}{|c|}{$16.00[\cos{\left(4.71\right)} + i\sin{\left(4.71\right)}]$}\\
	\hline
	\end{tabular}

    \item Quoziente algebrico (quinta funzione)\\
        \begin{tabular}{|*3{c|}}
	\hline
	\bfseries Dividendo		& \bfseries Divisore		& \bfseries Quoziente\\
	\hline
	$1.00 + 1.00 i$			& $1.00 - 1.00 i$		& $0.00 + 1.00 i$\\
	\hline
	\end{tabular}

    \item Quoziente trigonometrico (quinta funzione)\\
    	\begin{tabular}{|*2{c|}}
	\hline
	\bfseries Dividendo						& \bfseries Divisore\\
	\hline
	$2.00[\cos{\left(3.14\right)} + i\sin{\left(3.14\right)}]$	& $0$\\
	\hline
	\multicolumn{2}{|c|}{\textbf{Quoziente}}\\
	\hline
	\multicolumn{2}{|c|}{Impossibile calcolare (divisore = 0)}\\
	\hline
	\end{tabular}    	
\end{itemize}

\newpage

% nuova pagina - testing del programma
\section*{ \textbf{Verifica del programma} }


\end{document}
